\documentclass[conference]{IEEEtran}
\IEEEoverridecommandlockouts
% The preceding line is only needed to identify funding in the first footnote. If that is unneeded, please comment it out.
\usepackage{cite}
\usepackage{amsmath,amssymb,amsfonts}
\usepackage{algorithmic}
\usepackage{graphicx}
\usepackage{textcomp}
\usepackage{xcolor}
\usepackage{url}
\usepackage{hyperref}

\makeatletter
\newcommand{\linebreakand}{%
  \end{@IEEEauthorhalign}
  \hfill\mbox{}\par
  \mbox{}\hfill\begin{@IEEEauthorhalign}
}
\makeatother



\def\BibTeX{{\rm B\kern-.05em{\sc i\kern-.025em b}\kern-.08em
    T\kern-.1667em\lower.7ex\hbox{E}\kern-.125emX}}
\begin{document}

\title{¿Quién gasta más y quién se registra? Evidencia observacional en CheMarket}

\author{%
\IEEEauthorblockN{Adrián Arturo Suárez García}
\IEEEauthorblockA{202123771\\
\href{mailto:a.suarezg@uniandes.edu.co}{\texttt{a.suarezg@uniandes.edu.co}}}
\and
\IEEEauthorblockN{Luis Alejandro Rubiano Guerrero}
\IEEEauthorblockA{202013482\\
\href{mailto:la.rubiano@uniandes.edu.co}{\texttt{la.rubiano@uniandes.edu.co}}}
\and
\IEEEauthorblockN{Gabriel Alejandro Moreno Riveros}
\IEEEauthorblockA{202014583\\
\href{mailto:g.morenor@uniandes.edu.co}{\texttt{g.morenor@uniandes.edu.co}}}
\linebreakand
\IEEEauthorblockN{Juan Sebastián Sierra Tarazona}
\IEEEauthorblockA{202123725\\
\href{mailto:j.sierrat@uniandes.edu.co}{\texttt{j.sierrat@uniandes.edu.co}}}
}



\maketitle


\section{Introducción}
A través de este informe nuestro grupo de economistas y cientificos de datos 
emplea diferentes técnicas de análisis estadístico y machine learning
para estudiar el comportamiento de los usuarios en la \textit{CheMarket Inc}.
En particular, nos enfocaremos en entender qué variables generan revenue para la compañía. 
Este análisis ayuda a la toma de decisiones estratégicas informadas. 

\section{Datos observacionales: ¿Qué impulsa las ventas?}

\subsection{Datos y preparación}

Fuente y variables

Transformaciones

Partición

Tabla de variables 

Tabla estadísticas descriptivas

Matriz de correlaciones

\subsection{Estimación del efecto de registrarse}

Densidades

Histogramas

Boxplot

\subsection{Efecto de registrarse sobre el gasto}

Marco conceptual, sesgo por variables omitidas.

Diferentes modelos

pseudo coefplot coeficientes

Interpretación

multicolinealidad


\subsection{Reflexión sobre causalidad}

\subsection{Recomendación preliminar}


\section{Datos experimentales: ¿Funciona facilitar el registro?}

\subsection{Verificación del experimento}

\subsection{Efecto sobre el registro}

\subsection{Efecto sobre el gasto}

\subsection{Limitaciones y robustez}

\subsection{Recomendación final}

\section{Conclusiones}

Resumen 

Recomendaciones accionables.



\end{document}
